\documentclass[%
	ngerman,
	12pt,
	a4paper
]{scrbook}
\usepackage{graphicx}
\usepackage{babel}
\usepackage{minted}
\usepackage{hyperref}
\usepackage{fontspec}
\setmainfont{Source Sans Pro}
\setmonofont{Cascadia Code}
\usepackage{tcolorbox}

% Definition der Boxen
\newtcolorbox{tipbox}{
	colback=green!5!white,
	colframe=green!75!black,
	fonttitle=\bfseries,
	title=Tipp,
	sharp corners
}

\newtcolorbox{warningbox}{
	colback=red!5!white,
	colframe=red!75!black,
	fonttitle=\bfseries,
	title=Achtung,
	sharp corners
}

\newtcolorbox{examplebox}{
	colback=blue!5!white,
	colframe=blue!75!black,
	fonttitle=\bfseries,
	title=Beispiel,
	sharp corners
}

\title{Pflichtenheft}
\subtitle{Konzeption und prototypische Implementierung eines B2B-Webshops}
\author{WI22A-AKI}

\begin{document}
\maketitle

\section{Projektteilnehmer}
\begin{tabular}{l|l|l}
	\textbf{Name}                & \textbf{Vorname} & \textbf{Matrikelnummer} \\ \hline
	Christ (Projektleiter)       & Colin            & 4359760                 \\
	Spatzek (stv. Projektleiter) & Steffen          & 3854031                 \\ \hline
	Arnold                       & Daniel           & 8627710                 \\
	Bamberger                    & Bastian          & 2923282                 \\
	Denz                         & Andreas          & 5428962                 \\
	Jeevakanthan                 & Milan            & 9892846                 \\
	Kanjo                        & Alan             & 9795498                 \\
	Kunz                         & Paul             & 2338290                 \\
	Schreck                      & David            & 3533132                 \\
	Strohm                       & Julian           & 7956706                 \\
	Swoboda                      & Timo             & 4388948                 \\
	Tomanek                      & Lukas            & 5985858                 \\
	Väth                         & Luis             & 8122258                 \\
	Weis                         & Noah             & 1555500
\end{tabular}

\tableofcontents

\chapter{Zielsetzung}
Beschreibung des Projektziels:
\begin{itemize}
	\item Warum wird das Projekt durchgeführt?
	\item Welche Probleme sollen gelöst werden?
\end{itemize}

\section{Musskriterien}
	\subsection{Allgemeine Anforderungen}
	\begin{enumerate}
		\item Benutzerverwaltung

		Rollenbasierte Verwaltung:
		\begin{itemize}
			\item Administrator (alle Rechte, Pflege des Artikelkatalogs)
			\item Interne Mitarbeiter (eingeschränkte Rechte)
			\item Externe Kunden (Anlegen/Pflegen von Accounts, ggf. Sperren)
		\end{itemize}
		\item Artikelsuche und Anzeige
		\begin{itemize}
			\item Katalog zur Suche
			\item semantische Suche
		\end{itemize}
		\item Bestellprozess
		\begin{itemize}
			\item Implementierung eines rudimentären Prozesses für Bestellungen
		\end{itemize}
		\item KI-Komponente
		\begin{itemize}
			\item für Produktempfehlungen oder kundenspezifische Preisfestlegung
		\end{itemize}
		\item Aufgabenaufteilung
		\begin{itemize}
			\item Entwurf sinnvoller Use-Cases, Datenmodelle und Software-Architekturkomponenten
		\end{itemize}
		\item Meilensteine
		\begin{itemize}
			\item Definition von Projektmeilensteinen und Kommunikation bei Abweichungen
		\end{itemize}
		\item Software-Engineering-Prinzipien
		\begin{itemize}
			\item Anwendung von Kern- und Unterstützungsprozessen des Software-Engineering
			\item Einhaltung gängiger Namenskonventionen (z.~B. CamelCase, Methodennamen als Verben)
		\end{itemize}
		\item Dokumentation
		\begin{itemize}
			\item Angemessene und fortlaufende Dokumentation, einschließlich Programmcodes
		\end{itemize}
		\item Projektmanagement
		\begin{itemize}
			\item Regelmäßige Fortschrittsmitteilungen (z.~B. via E-Mail und in einem Log)
		\end{itemize}
	\end{enumerate}
	\subsection{Technische Anforderungen}
	\begin{enumerate}
		\item Frontend
		\begin{itemize}
			\item Web-Oberfläche basierend auf \texttt{HTML5/5.2}
			\item Optimierung für mobile Endgeräte (\textit{mobile first})
		\end{itemize}
		\item Backend
		\begin{itemize}
			\item Serverbasierte Implementierung (Java-Servlet/JSP oder PHP)
		\end{itemize}
		\item Datenbank
		\begin{itemize}
			\item Cloud-/Server-Anwendung auf Basis einer Datenbank
			\item Verwaltung und Nutzung ausschließlich über die Web-Oberfläche
		\end{itemize}
		\item Rollenmanagement
		\begin{itemize}
			\item Definition und Implementierung mehrerer sinnvoller Rollen
		\end{itemize}
	\end{enumerate}
	\subsection{Entwicklungsprozess}
	Agile Softwareentwicklung mit:
	\begin{itemize}
		\item Anforderungsanalyse
		\item Entwurf
		\item Implementierung
		\item Test und Dokumentation
	\end{itemize}
	\subsection{Präsentation und Abgabe}
	\begin{itemize}
		\item Präsentation des Prototyps im Plenum mit klarer Aufgabenverteilung innerhalb der Gruppe
		\item Abgabe eines funktionsfähigen Prototyps inklusive Dokumentation
	\end{itemize}

\section{Wunschkriterien}
	\subsection{Allgemeine Anforderungen}
	\begin{enumerate}
		\item Benutzerverwaltung
		\begin{enumerate}
			\item Individuelle Preisgestaltung für Kundengruppen
			\item mehrere Benutzerkonten pro Unternehmen (z.~B. je Abteilung)
			\item individuelle Dashboards mit relevanten Informationen
			\item automatische Rabatte für Stammkunden basierend auf Kaufhistorie
			\item Zwei-Faktor-Authentifizierung für erhöhte Sicherheit
		\end{enumerate}
		\item Artikelsuche und Anzeige
		\begin{enumerate}
			\item Erweiterte Filterfunktionen (z.~B. nach Energieeffizienz, Preis, Marke)
			\item Produktvergleichsfunktionen für technische Spezifikationen
		\end{enumerate}
		\item Bestellprozess
		\begin{enumerate}
			\item Unterstützung mehrerer Warenkörbe für verschiedene Projekte oder Abteilungen
			\item Automatisierung wiederkehrender Bestellungen durch Abo-Modelle
			\item Mengenrabatte, die automatisch im Warenkorb berechnet werden
		\end{enumerate}
		\item KI-Komponente
		\begin{enumerate}
			\item KI-gestützte Produktempfehlungen basierend auf Trends und Kaufhistorie
			\item KI-gestützter Chatbot für häufige Kundenanfragen (FAQ)
		\end{enumerate}
	\end{enumerate}
	\subsection{Technische Anforderungen}
	\begin{enumerate}
		\item Frontend
		\begin{enumerate}
			\item Dark-Mode-Unterstützung für augenschonende Bedienung
			\item Responsive Design für optimale Darstellung auf allen Endgeräten
			\item Drag-and-Drop-Funktion für die Verwaltung von Warenkörben
		\end{enumerate}
		\item Backend
		\begin{enumerate}
			\item API-Schnittstelle für ERP- und Einkaufssysteme
			\item Automatische Synchronisierung von Lagerbeständen
			\item Verwaltung individueller Preise und Rabatte
			\item Unterstützung zeitlich begrenzter Aktionen
			\item Logging- und Monitoring-Tools für Fehleranalyse
		\end{enumerate}
		\item Datenbank
		\begin{enumerate}
			\item Speicherung des Nutzerverhaltens für Produktempfehlungen
			\item Versionierung von Produktdaten für Nachvollziehbarkeit
			\item Volltextsuche für schnelle Suchergebnisse
			\item Automatisierte Backups und Wiederherstellung
		\end{enumerate}
		\item Rollenmanagement
		\begin{enumerate}
			\item Granulare Rechteverwaltung für Benutzerrollen
			\item Unterschiedliche Zugriffsrechte für Preise und Funktionen
			\item Rollenbasierte Anpassung sichtbarer Daten
			\item Audit-Logs zur Nachverfolgung von Aktionen
			\item Dynamische Erweiterung und Anpassung von Rollen
		\end{enumerate}
	\end{enumerate}
\section{Abgrenzungskriterien}
\begin{enumerate}
	\item Funktionale Abgrenzungen:
	\begin{enumerate}
		\item \textbf{Umfang des Produktangebots:} Der Shop beschränkt sich auf Hardwareprodukte, keine Dienstleistungen.
		\item \textbf{Kein Marktplatzmodell:} Der Shop dient nicht als Plattform für andere Anbieter.
	\end{enumerate}
	\item Technische Abgrenzungen:
	\begin{enumerate}
		\item \textbf{Keine mobile Anwendung:} Es wird keine App entwickelt. Der Shop soll als Webservice genutzt werden.
		\item \textbf{Keine Mehrsprachigkeit:} Der Shop wird ausschließlich in deutscher Sprache betrieben.
	\end{enumerate}
	\item Rechtliche Abgrenzungen:
	\begin{enumerate}
		\item \textbf{Keine rechtliche Anpassung für Nicht-EU-Länder:} Der Shop wird nicht an Steuer- und Rechtssysteme außerhalb der EU angepasst.
	\end{enumerate}
	\item Gestalterische Anpassung:
	\begin{enumerate}
		\item \textbf{Keine vollständige Barrierefreiheit:} Der Shop wird nicht vollständig barrierefrei entwickelt (z.~B. keine Optimierung für Screenreader oder spezielle Kontrasteinstellungen).
	\end{enumerate}
\end{enumerate}

\chapter{Produkteinsatz}
\section{Anwendungsbereich}
Der Anwendungsbereich des Webshops umfasst den Verkauf von IT-Hardware an Geschäftskunden.
Die Kunden erhalten Zugang zum Shop und können dort die benötigte Hardware bestellen.
\section{Zielgruppen}
Der B2B-Onlineshop für IT-Hardware richtet sich vor allem an drei Hauptzielgruppen:
IT-Dienstleister, Großunternehmen und Konzerne sowie Reseller.
\begin{itemize}
	\item IT-Dienstleister und Systemhäuser benötigen regelmäßig Hardware wie Server, Netzwerktechnik und Speichersysteme für den Aufbau und die Wartung von IT-Infrastrukturen bei ihren Kunden.
	Diese Zielgruppe verlangt große Bestellmengen, maßgeschneiderte Lösungen und eine zuverlässige Lieferung.
	\item Große Unternehmen und Konzerne beschaffen IT-Hardware für ihre Mitarbeiter und Abteilungen.
	Sie benötigen eine breite Produktpalette und einfache Bestellprozesse.
	\item Reseller hingegen kaufen IT-Produkte in großen Mengen, um sie weiterzuverkaufen.
	Sie benötigen wettbewerbsfähige Preise, detaillierte Produktinformationen sowie eine effiziente Bestell- und Lieferabwicklung.
\end{itemize}
\section{Betriebsbedingungen}
Die Anwendung läuft auf einem Webserver in einer eigenen containerisierten Docker-Umgebung.
Sie wird rund um die Uhr laufen, mit Ausnahme von Wartungsarbeiten.

\chapter{Produktübersicht}

\chapter{Detaillierte Produktfunktionen}
\section{User Stories}
\section{Funktionale Anforderungen}
\section{Nicht-funktionale Anforderungen}

\chapter{Produktdaten}
\chapter{Systemarchitektur}
\chapter{Datenmodell}
\chapter{Schnittstellendefinition (API)}
\chapter{Benutzungsoberflächen}

\chapter{Projektorganisation}
\section{Projektmethodik}
\section{Rollenverteilung}

\end{document}