\documentclass[%
	12pt,
	a4paper,
	oneside,
	parskip=full
]{scrbook}
\usepackage{graphicx}
\usepackage{polyglossia}
\setdefaultlanguage{german}
\usepackage{minted}
\usepackage{hyperref}
\usepackage{fontspec}
\setmainfont{xcharter}
\usepackage{tcolorbox}
\usepackage{csquotes}
\usepackage{setspace}
\usepackage{cleveref}
\usepackage{xcolor}
\usepackage{listings}
\usepackage{longtable}
\usepackage{enumitem}
\usepackage{tabularx}

\definecolor{lightgray}{gray}{0.95}

\lstdefinelanguage{json}{
	basicstyle=\ttfamily\small,
	numbers=left,
	numberstyle=\scriptsize,
	stepnumber=1,
	numbersep=8pt,
	backgroundcolor=\color{lightgray},
	showspaces=false,
	showstringspaces=false,
	breaklines=true,   % Automatischer Zeilenumbruch
	frame=single,      % Rahmen um Listings
	captionpos=b       % Caption unter dem Listing
}

% Definition der Boxen
\newtcolorbox{tipbox}{
	colback=green!5!white,
	colframe=green!75!black,
	fonttitle=\bfseries,
	title=Tipp,
	sharp corners
}

\newtcolorbox{warningbox}{
	colback=red!5!white,
	colframe=red!75!black,
	fonttitle=\bfseries,
	title=Achtung,
	sharp corners
}

\newtcolorbox{examplebox}{
	colback=blue!5!white,
	colframe=blue!75!black,
	fonttitle=\bfseries,
	title=Beispiel,
	sharp corners
}

\title{Pflichtenheft}
\subtitle{Konzeption und prototypische Implementierung eines B2B-Webshops}


\author{WI22A-AKI}


\begin{document}
%\maketitle

\begin{titlepage}
	\centering
	% Logo hinzufügen
	\includegraphics[width=0.5\textwidth]{Media/dhbwlogo.png}\vspace{1cm}

	{\large Projektkonzeption und -realisierung\par}
	\vspace{0.5cm}
	{\Large\bfseries Konzeption und prototypische Implementierung eines B2B-Webshops\par}
	\vspace{1.5cm}

%	{\Huge \texttt{\MakeUppercase{Pflichtenheft}}}\par
	{\Huge \textbf{Pflichtenheft}\par}
	\vspace{2cm}

	\doublespacing
	{\large \textbf{Wintersemester 2024/2025}\par}
	\textbf{Semester:} 5\\
	\textbf{Kurs:} WI22A-AKI

	{\large\textbf{Betreuer:}\\}
	{\large Prof. Dr. Alexandros Nanopoulos} \\
	{\large Prof. Dr. Dirk Palleduhn}

	\vfill

	Mosbach, den \today  %TODO

\end{titlepage}


\section{Projektteilnehmer}
\begin{table}[ht]
	\centering
	\begin{tabular}{l|l|l}
		\textbf{Name}                & \textbf{Vorname} & \textbf{Matrikelnummer} \\ \hline
		Christ (Projektleiter)       & Colin            & 4359760                 \\
		Spatzek (stv. Projektleiter) & Steffen          & 3854031                 \\ \hline
		Arnold                       & Daniel           & 8627710                 \\
		Bamberger                    & Bastian          & 2923282                 \\
		Denz                         & Andreas          & 5428962                 \\
		Jeevakanthan                 & Milan            & 9892846                 \\
		Kanjo                        & Alan             & 9795498                 \\
		Kunz                         & Paul             & 2338290                 \\
		Schreck                      & David            & 3533132                 \\
		Strohm                       & Julian           & 7956706                 \\
		Swoboda                      & Timo             & 4388948                 \\
		Tomanek                      & Lukas            & 5985858                 \\
		Väth                         & Luis             & 8122258                 \\
		Weis                         & Noah             & 1555500
	\end{tabular}
	\caption{Projektteilnehmer und Matrikelnummern}
	\label{tab:projektteilnehmer}
\end{table}

\section{Organigramm}
Dieses Kapitel beschreibt die Organisationsstruktur des Projekts \textit{B2B-Webshop} und stellt die Rollen und Verantwortlichkeiten der Teammitglieder in einem übersichtlichen Organigramm (siehe \cref{fig:organigramm}) dar.
Dabei wird zwischen dem Projektleiter und dem Co-Projektleiter unterschieden, die jeweils eigene Teams aus Entwicklern und Fachkräften führen.

Das Organigramm dient als visuelle Orientierungshilfe, um die Aufgabenverteilung und Kommunikationswege im Projekt klar darzustellen.
Ziel ist es, die Zusammenarbeit zu fördern und sicherzustellen, dass alle Beteiligten ihre Verantwortlichkeiten kennen.
\begin{figure}[ht]
	\centering
	\includegraphics[width=0.4\linewidth]{"out/Organigramm B2B-Webshop"}
	\caption{Organigramm der Projektgruppe des B2B-Webshops}
	\label{fig:organigramm}
\end{figure}

\section{Teamstruktur}
Das Kapitel Teamstruktur gibt einen tabellarischen Überblick über die Zuordnung der Teammitglieder zu den zentralen Verantwortungsbereichen des Projekts.
Die Rollen sind in die Kategorien Organisation, Entwicklung, Solution Architect und User Experience unterteilt (siehe \cref{tab:teamstruktur}).

Die Tabelle verdeutlicht die Zuordnung der Teammitglieder zu ihren jeweiligen Verantwortungsbereichen und unterstreicht die multidisziplinäre Zusammenarbeit innerhalb des Projekts.
Die Hervorhebung einzelner Mitglieder zeigt deren Führungs- oder Spezialaufgaben, während die klare Struktur die Nachvollziehbarkeit und Effizienz innerhalb des Teams unterstützt.
\begin{table}[ht]
	\centering
	\begin{tabular}{l|l|l|l}
		 \textbf{Organisation}   &    \textbf{Entwicklung}     & \textbf{Solution Architect} & \textbf{User Experience} \\ \hline
		\textcolor{purple}{Colin Christ} & \textcolor{purple}{Steffen Spatzek} & \textcolor{purple}{Steffen Spatzek} &        Alan Kanjo        \\ \hline
		     Lukas Tomanek       &          Paul Kunz          &         Alan Kanjo          &      Lukas Tomanek       \\ \hline
		     David Schreck       &        Timo Swoboda         &          Paul Kunz          &      David Schreck       \\ \hline
		   Bastian Bamberger     &         Alan Kanjo          &          Luis Väth          &      Julian Strohm       \\ \hline
		   Milan Jeevakanthan    &        Julian Strohm        &                             &    Bastian Bamberger     \\ \hline
		       Luis Väth         &        Andreas Denz         &                             &    Milan Jeevakanthan    \\ \hline
		     Daniel Arnold       &          Noah Weis          &                             &
	\end{tabular}
	\caption{Verantwortungsbereiche der Projektmitglieder des B2B-Webshops}
	\label{tab:teamstruktur}
\end{table}

\tableofcontents

\chapter{Zielsetzung}
\begin{figure}[ht]
	\centering
	\includegraphics[width=0.5\linewidth]{"Media/Hot Hardware Hub Logo"}
	\caption[Logo Hot Hardware Hub]{Firmenlogo des Hot Hardware Hub}
	\label{fig:hot-hardware-hub-logo}
\end{figure}
Der \textbf{Hot Hardware Hub} ist ein fiktives Unternehmen, das im Rahmen dieses Projektes gegründet wurde, um hochwertige und moderne IT-Hardware speziell für Geschäftskunden (B2B) anzubieten.
Ziel ist es, einen innovativen Webshop zu entwickeln, der es Unternehmen ermöglicht, die benötigte Hardware schnell, einfach und bequem online zu bestellen.

Im Mittelpunkt des Systems steht das Ziel, den Kunden ein erstklassiges Einkaufserlebnis zu bieten.
Eine intuitive Benutzeroberfläche sowie ein klar strukturierter und gut durchsuchbarer Produktkatalog ermöglichen es den Nutzern, mit nur wenigen Klicks die passenden Produkte zu finden und die Bestellungen mühelos abzuschließen.

Die Systementwicklung reagiert auf die steigende Nachfrage nach digitalen Beschaffungslösungen im IT-Bereich.
Viele Unternehmen suchen nach effizienten Möglichkeiten, schnell und unkompliziert hochwertige Hardware zu beschaffen.
Der Webshop des \textbf{Hot Hardware Hub} stellt hierfür eine zuverlässige und benutzerfreundliche Plattform bereit, die die IT-Beschaffung deutlich vereinfacht.
Das Ziel ist es, nicht nur Zeit und Aufwand zu sparen, sondern auch die Zufriedenheit und Effizienz der Geschäftskunden nachhaltig zu steigern.

\section{Musskriterien}
	\vspace{0.5cm}
	\textbf{Produktkatalog}
	\begin{itemize}
		\item Kunden können den Produktkatalog mit der IT-Hardware einsehen.
		\item Produkte werden in Kategorien darstellbar angezeigt.
		\item Produkte müssen die wesentlichen Merkmale (Preis, Menge, Lagerbestand und Produktdetails) für den Kunden sichtbar machen.
		\item Produkte können über eine Suchfunktion mit Filtermöglichkeiten gezielt gefiltert werden.
	\end{itemize}

	\vspace{0.5cm}
	\textbf{Benutzerverwaltung}
	\begin{itemize}
		\item Kunden müssen sich registrieren und sich mit ihren Daten anmelden können.
		\item Kunden müssen ihre Benutzerdaten einsehen und verändern können.
		\item Kunden müssen ihr Konto deaktivieren bzw. löschen können.
		\item Der Anmeldeprozess im Webshop muss mit einer sicheren Authentifizierungsmethode gestaltet werden.
	\end{itemize}

	\vspace{0.5cm}
	\textbf{Bestellprozess}
	\begin{itemize}
		\item Kunden müssen ihre Produkte in den Warenkorb legen und diesen einsehen können.
		\item Der Kunde muss eine Bestellübersicht vor dem finalen Abschluss sehen.
		\item Der Kunde muss in minimalen Schritten zum erfolgreichen Kaufabschluss geführt werden.
		\item Kunden müssen im Kundenbereich getätigte Bestellungen und deren Status einsehen können.
	\end{itemize}

	\vspace{0.5cm}
	\textbf{Zahlung und Rechnungsstellung}
	\begin{itemize}
		\item Dem Kunden müssen gängige Zahlungsmethoden verfügbar gemacht werden (z.~B. Kauf auf Rechnung).
		\item Der Kunde muss nach erfolgreichem Abschluss eine Rechnung per E-Mail erhalten oder diese im Kundenbereich einsehen können.
	\end{itemize}

	\vspace{0.5cm}
	\textbf{Shop-Betreiber}
	\begin{itemize}
		\item Administratoren können über ein Dashboard Produkte und zugehörige Daten erstellen, bearbeiten und löschen.
		\item Der Shop-Betreiber kann Produktbilder und Dokumente mit den Produktseiten verknüpfen.
	\end{itemize}

	\vspace{0.5cm}
	\textbf{KI-Komponente}
	\begin{itemize}
		\item Es soll eine KI-Komponente integriert werden, die den Kunden per Chat beim Einkauf unterstützt.
		\item Über eine ML-Komponente soll erreicht werden, dass Kundenpreise je nach Einkaufsvolumen bzw. -verhalten individuell rabattiert werden.
	\end{itemize}

	\vspace{0.5cm}
	\textbf{Technische Aspekte}
	\begin{itemize}
		\item Der Webshop soll plattformunabhängig von den gängigsten Geräten aufgerufen werden können.
		\item Der Webshop soll Anfragen schnell abarbeiten und schnell erreichbar sein.
		\item Die Webapplikation soll eine intuitive Bedienung aufweisen.
		\item Der B2B-Shop soll durch steigende Produktmengen schnell skalierbar sein.
	\end{itemize}

\section{Wunschkriterien}
	\vspace{0.5cm}
	\textbf{Benutzerverwaltung}
	\begin{itemize}
		\item Es soll ermöglicht werden, dass eine eigene Einkaufsgruppe für einen gewissen Kundenkreis erstellt werden kann.
		\item Der Kunde soll mehr als nur einen Warenkorb anlegen, befüllen und speichern können.
		\item Mehrere Benutzerkonten oder -gruppen für ein Unternehmen sollen unterstützt werden.
		\item Es sollen Kennzahlen für einen bestimmten Kunden oder eine Einkaufsgemeinschaft bereitgestellt werden.
		\item Zwei-Faktor-Authentifizierung oder eine No-Password-Authentication (z.~B. über Passkeys) soll dem Kunden ermöglicht werden.
	\end{itemize}

	\vspace{0.5cm}
	\textbf{Produktkatalog}
	\begin{itemize}
		\item Kundenbenachrichtigungen bei wieder verfügbaren Artikeln.
		\item Eine noch detailliertere Filterfunktion bei der Produktsuche.
		\item Produktvergleich-Funktion zwischen zwei oder mehreren Produkten.
		\item Zeitlich begrenzte Aktionen oder individuelle Gutscheine.
	\end{itemize}

	\vspace{0.5cm}
	\textbf{Sicherheitsaspekte}
	\begin{itemize}
		\item Das System muss alle Aspekte der DSGVO erfüllen.
		\item Es ist eine dem Stand der Technik entsprechende Datenverschlüsselung zu verwenden.
	\end{itemize}

	\vspace{0.5cm}
	\textbf{Bestellprozess}
	\begin{itemize}
		\item Ein wiederkehrendes Bestellmodell soll angeboten werden.
		\item Individuelle Mengenrabatte je nach Menge oder Einkaufsvolumen in einem bestimmten definierten Zeitraum.
	\end{itemize}

	\vspace{0.5cm}
	\textbf{KI-Komponente}
	\begin{itemize}
		\item Auf Basis von Wunschlisten, Kaufhistorie oder neuen Artikeln im Sortiment werden KI-gestützte Produktempfehlungen gegeben.
	\end{itemize}

	\vspace{0.5cm}
	\textbf{Technische Aspekte}
	\begin{itemize}
		\item Darstellung von Daten wie Traffic, Besucherzahlen und Kundenaktionen in einem Dashboard für den Shopbetreiber.
		\item Logging und Monitoring des Webshops.
	\end{itemize}


\section{Abgrenzungskriterien}
\begin{enumerate}
	\item Funktionale Abgrenzungen:
	\begin{enumerate}
		\item \textbf{Umfang des Produktangebots:} Der Shop beschränkt sich auf Hardwareprodukte, keine Dienstleistungen.
		\item \textbf{Kein Marktplatzmodell:} Der Shop dient nicht als Plattform für andere Anbieter.
	\end{enumerate}
	\item Technische Abgrenzungen:
	\begin{enumerate}
		\item \textbf{Keine mobile Anwendung:} Es wird keine App entwickelt. Der Shop soll als Webservice genutzt werden.
		\item \textbf{Keine Mehrsprachigkeit:} Der Shop wird ausschließlich in deutscher Sprache betrieben.
	\end{enumerate}
	\item Rechtliche Abgrenzungen:
	\begin{enumerate}
		\item \textbf{Keine rechtliche Anpassung für Nicht-EU-Länder:} Der Shop wird nicht an Steuer- und Rechtssysteme außerhalb der EU angepasst.
	\end{enumerate}
	\item Gestalterische Anpassung:
	\begin{enumerate}
		\item \textbf{Keine vollständige Barrierefreiheit:} Der Shop wird nicht vollständig barrierefrei entwickelt (z.~B. keine Optimierung für Screenreader oder spezielle Kontrasteinstellungen).
	\end{enumerate}
\end{enumerate}

\chapter{Produkteinsatz}
\section{Anwendungsbereich}
Der Anwendungsbereich des Webshops umfasst den Verkauf von IT-Hardware an Geschäftskunden.
Die Kunden erhalten Zugang zum Shop und können dort die benötigte Hardware bestellen.
\section{Zielgruppen}
Der B2B-Onlineshop für IT-Hardware richtet sich vor allem an drei Hauptzielgruppen:
IT-Dienstleister, Großunternehmen und Konzerne sowie Wiederverkäufer:
\begin{itemize}
	\item IT-Dienstleister und Systemhäuser benötigen regelmäßig Hardware wie Server, Netzwerktechnik und Speichersysteme für den Aufbau und die Wartung von IT-Infrastrukturen bei ihren Kunden.
	Diese Zielgruppe verlangt nach großen Bestellmengen, maßgeschneiderten Lösungen und zuverlässiger Lieferung.
	\item Große Unternehmen und Konzerne beschaffen IT-Hardware für ihre Mitarbeiter und Abteilungen.
	Sie benötigen eine breite Produktpalette und einfache Bestellprozesse.
	\item Reseller (Wiederverkäufer) hingegen kaufen IT-Produkte in großen Mengen ein, um sie weiterzuverkaufen.
	Sie benötigen wettbewerbsfähige Preise, detaillierte Produktinformationen sowie eine effiziente Bestell- und Lieferabwicklung.
\end{itemize}
\section{Betriebsbedingungen}
Die Anwendung läuft auf einem Webserver in einer eigenen containerisierten Docker-Umgebung.
Sie läuft rund um die Uhr, mit Ausnahme von Wartungsfenstern.

\chapter{Produktübersicht}
\begin{figure}[ht]
	\centering
	\includegraphics[width=\textwidth]{project_models_diagrams/use_case_diagramm}
	\caption{Use Case-Diagramm}
\end{figure}
Für das Produkt \emph{Hot Hardware Hub} sind drei Akteure vorgesehen:
\begin{description}
	\item[Admin] Verwaltung und Pflege des Kundenstamms und des Produktkatalogs
	\item[Kunde] Verantwortlicher Einkäufer des Kunden, der weitere Mitarbeiter für seinen Einkäuferstamm berechtigen kann
	\item[Subkunde] Berechtiger Einkäufer eines Kunden, der Produkte im \emph{Hot Hardware Hub} einkauft
\end{description}

\chapter{Detaillierte Produktfunktionen}
\begin{enumerate}[label=F\arabic*]
    \item Anmeldung
    \begin{enumerate}[label=F\arabic{enumi}.\arabic*]
        \item Kunde kann sich registrieren
        \item Kunde kann sich nach Registrierung anmelden
        \item Kunde kann seine Benutzerdaten anzeigen
        \item Kunde kann gezielt ein Produkt in der Detailansicht öffnen
    \end{enumerate}
    
    \item Anpassung Benutzerdaten (Subkunde)
    \begin{enumerate}[label=F\arabic{enumi}.\arabic*]
        \item Subkunde kann Benutzerdaten ändern
        \item Subkunde kann Passwort zurücksetzen
        \item Subkunde kann Benutzerkonto löschen
    \end{enumerate}
    
    \item Admin-Benutzer
    \begin{enumerate}[label=F\arabic{enumi}.\arabic*]
        \item Adminkonto registrieren
        \item Admindashboard anzeigen
        \item Produkte zu CMS hinzufügen
        \item Produktdetails pflegen
        \item Erweiterung Item-Model
    \end{enumerate}
    
    \item Produktkatalog
    \begin{enumerate}[label=F\arabic{enumi}.\arabic*]
        \item Kunde kann Produktkatalog ansehen
        \item Kunde kann im Produktkatalog nach Suchkriterien suchen
        \item Kunde kann per Filterfunktion gewisse Artikel ausblenden/einblenden
        \item Kunde kann gezielt ein Produkt in der Detailansicht öffnen
    \end{enumerate}
\end{enumerate}

\section{User Stories}
An dieser Stelle sei auf das Scrum-Board verwiesen, das unter folgendem Link zu finden ist: \\
\url{https://tree.taiga.io/project/ssptzk-b2b-webshop}

Damit werden Redundanzen in der Dokumentation vermieden.

\section{Funktionale Anforderungen}
Die funktionalen Anforderungen des Webshops orientieren sich an gängigen Geschäftsfällen, um die Prozesse der IT-Beschaffung effizient abzubilden. Dazu zählen unter anderem:
\begin{itemize}
	\item \textbf{Benutzerverwaltung:} Registrierung, Anmeldung und Verwaltung von Geschäftskunden sowie Unterstützung von Einkaufsgruppen mit Freigabeprozessen.
	\item \textbf{Produktverwaltung:} Hinzufügen, Bearbeiten und Entfernen von Produkten durch Administratoren.
	\item \textbf{Warenkorb- und Bestellprozess:} Produkte in den Warenkorb legen, Bestellungen aufgeben und Bestellhistorien einsehen.
	\item \textbf{Such- und Filterfunktionen:} Detaillierte Suche und Filterung nach Produktkategorien, Eigenschaften und Verfügbarkeit.
	\item \textbf{E-Mail-Benachrichtigungen:} Automatisierte E-Mails für Bestellbestätigungen, Rückmeldungen und Freigabeprozesse.
\end{itemize}

Diese Anforderungen decken die zentralen Funktionen ab, die für einen effizienten Betrieb des Webshops und die Erfüllung der Geschäftsziele notwendig sind.

\chapter{Datenstruktur und verwaltete Informationen}
Dieses Kapitel beschreibt die strukturierten Daten, die im Webshop verarbeitet und verwaltet werden. Dazu gehören Produktdaten, Bestandsinformationen sowie Kundendaten, die für die Geschäftsprozesse relevant sind.

\section{Produktdaten}
Die folgenden Informationen werden für jedes Produkt im Webshop gespeichert:

\begin{itemize}
	\item \textbf{Bezeichnung:} Produktname (z. B. \textit{Dell Latitude 5520})
	\item \textbf{Kurze Beschreibung:} Prägnante Angabe des Produkts (z. B. \textit{Hochwertiger Business-Laptop mit Intel i7})
	\item \textbf{Ausführliche Beschreibung:}  
	Detaillierte Produktspezifikationen, z. B.:  
	\textit{Intel i7-1165G7, 16GB RAM, 512GB SSD, 15,6-Zoll-Display, Windows 11 Pro, geeignet für anspruchsvolle Geschäftsanwendungen.}
	\item \textbf{Tatsächlicher Verkaufspreis:} 1.450 €  
	\item \textbf{Herstellerinformationen:}  
	\begin{itemize}
		\item Herstellername (z. B. \textit{Dell})  
		\item Modellnummer (z. B. \textit{Latitude 5520})  
	\end{itemize}
\end{itemize}

\section{Bestandsdaten}
Folgende Bestandsinformationen werden verwaltet:

\begin{itemize}
	\item \textbf{Verfügbare Anzahl im Lager:} 150 Einheiten von Artikel X
	\item \textbf{Produkte im Warenkorb:} 10 Artikel X in 5 verschiedenen Warenkörben
	\item \textbf{Bestellte, aber noch nicht gelieferte Artikel:} 20 Artikel X sind in ausstehenden Lieferungen in Bearbeitung
\end{itemize}

\section{Kundendaten}
Für registrierte Geschäftskunden werden die folgenden Daten gespeichert:

\begin{itemize}
	\item \textbf{Name und Firmenname:} Jürgen IT GmbH
	\item \textbf{Liefer- und Rechnungsadresse:} Hauptstraße 15, 74821 Mosbach
	\item \textbf{E-Mail-Adresse:} juergenIT@web.de
	\item \textbf{Bisheriges Einkaufsvolumen:} 25.000 €
	\item \textbf{Aktueller Warenkorb:} 3 Artikel (z. B. 2 Monitore und 1 Drucker)
	\item \textbf{Einkaufsgruppen und Freigabeprozesse:}  
	Mitarbeiter mit entsprechender Berechtigung haben Zugriff auf den Bestellverlauf und die Bestellliste.
\end{itemize}

\chapter{Qualitätsanforderungen}
In diesem Kapitel werden die Qualitätsmerkmale des zu entwickelnden Produkts und deren erforderliche Qualitätsstufen definiert. Die Qualitätsmerkmale müssen in operationalisierter Form vorliegen.

\begin{enumerate}
	\item \textbf{Benutzbarkeit (Verständlichkeit)} \\
	Die Benutzeroberfläche des Webshops muss klar strukturiert und intuitiv verständlich sein. Geschäftskunden sollen ohne lange Einarbeitung schnell die benötigte Hardware finden und bestellen können. Dies erfordert eine durchdachte Navigation, verständliche Produktbeschreibungen und eine optimierte Checkout-Führung.  
	Der Webshop muss eine konsistente UI/UX-Gestaltung aufweisen und mit gängigen B2B-Shop-Systemen vergleichbar sein, um die Benutzerfreundlichkeit zu maximieren.
	
	\item \textbf{Effizienz (Zeitverhalten)} \\
	Da Geschäftskunden effiziente Beschaffungslösungen benötigen, muss der Webshop schnelle Ladezeiten und eine reibungslose Performance gewährleisten.  
	- Produktseiten, Suchanfragen und der Checkout-Prozess sollen innerhalb von \textbf{maximal zwei Sekunden} geladen werden.  
	- Verzögerungen beim Einkaufsprozess sind zu minimieren, um Kaufabbrüche zu vermeiden.  
	- Ein Caching-Mechanismus und eine performante Datenbankanbindung sind erforderlich, um Lastspitzen abzufangen.
	
	\item \textbf{Interoperabilität} \\
	Die angebotene Hardware muss nahtlos in bestehende IT-Infrastrukturen der Geschäftskunden integrierbar sein.  
	- Der Webshop muss erweiterte Filter- und Suchfunktionen bieten, um eine gezielte Produktauswahl zu ermöglichen.  
	- Kunden sollen nach Betriebssystem, Anschlüssen, Dockingstation-Kompatibilität und weiteren relevanten Kriterien filtern können.  
	- Eine API-Schnittstelle soll bereitgestellt werden, um eine Anbindung an externe ERP-Systeme zu ermöglichen.
	
	\item \textbf{Übertragbarkeit (Installierbarkeit)} \\
	IT-Hardware für Unternehmen muss oft einfach zu installieren sein, insbesondere Server, Netzwerktechnik oder Arbeitsplatzlösungen.  
	- Der Webshop muss umfassende technische Spezifikationen, Treiberinformationen und Installationsanleitungen bereitstellen.  
	- Plug-and-Play-Funktionalitäten sollten klar ausgewiesen werden.  
	- Kompatibilitätsinformationen mit anderen Systemen sind transparent darzustellen.
	
	\item \textbf{Zuverlässigkeit (Wiederherstellbarkeit)} \\
	Die IT-Beschaffung muss auch bei Problemen reibungslos funktionieren.  
	- Der Webshop muss Rückgabeoptionen, Garantiebedingungen und Service-Level-Agreements (SLAs) klar und verständlich kommunizieren.  
	- Ein robustes Bestell- und Transaktionsmanagementsystem soll sicherstellen, dass Bestellungen auch im Falle technischer Probleme nicht verloren gehen.  
	- Im Fehlerfall muss eine automatische Wiederherstellung von abgebrochenen Bestellprozessen gewährleistet sein.  
\end{enumerate}

\chapter{Systemarchitektur}
Die nachfolgende Abbildung zeigt die Systemarchitektur des Webshops für den \textit{Hot Hardware Hub}. Der gesamte Aufbau basiert auf einer containerisierten Umgebung, die mithilfe von \textit{Dokploy} auf einem VPS bereitgestellt wird. Die Webshop-Komponenten, Datenbank sowie der Mailservice werden in einem separaten Container ausgeführt, um eine klare Trennung der Komponenten und eine bessere Skalierbarkeit zu gewährleisten.

\section{Komponenten der Architektur}
\begin{enumerate}
	\item \textbf{Traefik als Reverse Proxy und Loadbalancer} \\
	Traefik dient als zentrale Schnittstelle für den Datenverkehr zwischen den Clients und dem System. Es übernimmt die Verteilung der Anfragen an die entsprechenden Komponenten und ermöglicht eine sichere Kommunikation.
	
	\item \textbf{Frontend – Angular und NGINX} \\
	Die Benutzeroberfläche des Webshops wurde mit Angular realisiert. Diese wird über einen NGINX-Webserver ausgeliefert, der statische Inhalte wie HTML, CSS und JavaScript effizient bereitstellt.
	
	\item \textbf{Backend – Python Django Framework} \\
	Das Backend basiert auf dem Python-Django-Framework und wird durch einen \textit{Uvicorn}-Server bereitgestellt. Django stellt die REST-API bereit, über die das Frontend mit dem Backend kommuniziert. Die API wird durch einen dedizierten API-Gateway abgesichert, das die Anfragen verwaltet und gegebenenfalls weiterleitet.
	
	\item \textbf{Datenbank – PostgreSQL} \\
	Die relationalen Daten des Webshops werden in einer PostgreSQL-Datenbank gespeichert. Diese bietet eine zuverlässige und skalierbare Lösung für die Verwaltung der Geschäftsdaten wie Produkte, Bestellungen und Kundendaten.
	
	\item \textbf{Statische Ressourcen und Testdaten} \\
	Statische Inhalte wie Medienoder andere statische Ressourcen werden zentral gespeichert und von Django bereitgestellt.
	
	\item \textbf{E-Mail-Service – Go-Mailservice} \\
	Für die Abwicklung von Benachrichtigungen und E-Mails wird ein separater Go-Mailservice verwendet. Dieser ist über eine REST-Schnittstelle in die Systemarchitektur integriert und stellt sicher, dass alle E-Mails zuverlässig und performant versendet werden.
\end{enumerate}

\section{Containerisierung und Deployment}
Alle genannten Dienste werden in Docker-Containern ausgeführt, was eine portable und konsistente Umgebung gewährleistet. Durch den Einsatz von \textit{Dokploy} wird das Deployment automatisiert und in verschiedene Umgebungen wie Test- und Produktionssysteme orchestriert. Diese Containerisierung sorgt für eine klare Trennung der Dienste und ermöglicht es, Updates oder Skalierungen gezielt durchzuführen, ohne das Gesamtsystem zu beeinträchtigen.

\begin{figure}[ht]
	\centering
	\includegraphics[width=\textwidth, angle=90]{project_models_diagrams/system_architecture.png}
	\caption{Systemarchitektur}
\end{figure}


\chapter{Datenmodell}
\begin{itemize}
	\item Beschreibung Datenstruktur und deren Beziehung (ER-Diagramm erstellen)
	\item Datenbanktabellen beschreiben, Felder der Tables
\end{itemize}

\chapter{Schnittstellendefinition (API)}

%\begin{longtable}{|l|l|p{6cm}|p{6cm}|}
\begin{longtable}{|l|l|l|l|}
	\hline
	\textbf{Methode} & \textbf{API-Endpunkt} & \textbf{Übergebene Daten (Request)} & \textbf{Antwort (Response)} \\
	\hline
	POST & /web/api/auth/register/ & \ref{lst:register-request} & \ref{lst:register-response} \\
	\hline
	POST & /web/api/auth/login/ & \ref{lst:login-request} & \ref{lst:login-response} \\
	\hline
	POST & /web/api/auth/refresh/ & \ref{lst:refresh-request} & \ref{lst:refresh-response} \\
	\hline
	GET & /web/api/me/profile/ & - & \ref{lst:profile-response} \\
	\hline
	POST & /web/api/me/orders/ & \ref{lst:order-request} & \ref{lst:order-response} \\
	\hline
	\caption{API-Schnittstellendokumentation des HotHardwareHub-Shops}
	\label{tab:api_docs}
\end{longtable}

\begin{lstlisting}[language=json, caption={Request für Registrierung}, label=lst:register-request]
	{
		"email": "test@example.com",
		"password": "securepassword",
		"password_confirm": "securepassword"
	}
\end{lstlisting}

\begin{lstlisting}[language=json, caption={Response für Registrierung}, label=lst:register-response]
	{
		"email": "test@example.com"
	}
\end{lstlisting}

\begin{lstlisting}[language=json, caption={Request für Login}, label=lst:login-request]
	{
		"email": "test@example.com",
		"password": "securepassword"
	}
\end{lstlisting}

\begin{lstlisting}[language=json, caption={Response für Login}, label=lst:login-response]
	{
		"access": "JWT_ACCESS_TOKEN",
		"refresh": "JWT_REFRESH_TOKEN"
	}
\end{lstlisting}

\begin{lstlisting}[language=json, caption={Request für Token Refresh}, label=lst:refresh-request]
	{
		"refresh": "JWT_REFRESH_TOKEN"
	}
\end{lstlisting}

\begin{lstlisting}[language=json, caption={Response für Token Refresh}, label=lst:refresh-response]
	{
		"access": "NEW_JWT_ACCESS_TOKEN"
	}
\end{lstlisting}

\begin{lstlisting}[language=json, caption={Response für Profilabruf}, label=lst:profile-response]
	{
		"email": "test@example.com",
		"company_identifier": "12345",
		"company_name": "Tech Corp"
	}
\end{lstlisting}

\begin{lstlisting}[language=json, caption={Request für Bestellungen}, label=lst:order-request]
	{
		"order_info": {
			"buyer_name": "Max Mustermann"
		},
		"items": [
		{ "item_id": 1, "quantity": 2 }
		]
	}
\end{lstlisting}

\begin{lstlisting}[language=json, caption={Response für Bestellungen}, label=lst:order-response]
	{
		"order_id": 11,
		"order_status": "pending",
		"order_total": 49.99
	}
\end{lstlisting}

Die vollständige Liste aller verfügbaren Endpunkte, inklusive detaillierter Dokumentation und interaktiver Testmöglichkeiten, kann über die \textbf{Swagger-UI} abgerufen werden. Diese stellt eine umfassende Übersicht bereit und ermöglicht es Entwicklern, die API effizient zu nutzen und Endpunkte direkt zu testen.

\url{http://hothardwarehubtest-shop-xco4pb-80b302-5-75-130-54.traefik.me/web/api/swagger/}

\chapter{Benutzungsoberflächen}
Dieses Kapitel beschreibt die Gestaltung der Benutzungsoberflächen des Webshops, einschließlich der zugrunde liegenden Richtlinien und Designentscheidungen. Ziel war es, eine benutzerfreundliche, ästhetische und funktionale Oberfläche zu schaffen, die den Anforderungen von Geschäftskunden gerecht wird.

\section{Gestaltungsrichtlinien}
Die Entwicklung der Benutzeroberflächen erfolgte auf Grundlage der folgenden Prinzipien:
\begin{itemize}
	\item \textbf{Klarheit und Übersichtlichkeit:} Die Benutzeroberfläche wurde bewusst schlicht und übersichtlich gestaltet, damit sich Nutzer schnell orientieren können.
	\item \textbf{Konsistenz:} Einheitliche Farben, Schriften und Layout-Elemente sorgen für ein durchgängiges Erscheinungsbild und intuitive Bedienbarkeit.
	\item \textbf{Fokus auf Funktionalität:} Alle Designelemente unterstützen die Nutzbarkeit und wurden auf unnötige Komplexität verzichtet.
	\item \textbf{Responsivität:} Die Benutzeroberfläche ist für verschiedene Endgeräte optimiert, sodass der Webshop sowohl auf Desktops als auch auf mobilen Geräten problemlos genutzt werden kann.
\end{itemize}

\section{Verwendete Farben, Schriften und Layout}
\subsection{Farben}
Das Farbschema des Webshops basiert auf den Farben des erstellten Logos, um ein einheitliches und professionelles Erscheinungsbild zu gewährleisten. Die Logo-Farben wurden gezielt auf die Benutzeroberfläche übertragen, um eine visuelle Wiedererkennung und Markenidentität zu schaffen.

\begin{itemize}
	\item \textbf{Primärfarben:} Die Hauptfarben des Logos werden verwendet, um wichtige Elemente wie Buttons, Links und Header hervorzuheben.
	\item \textbf{Sekundärfarben:} Ergänzende Farben aus der Farbpalette des Logos dienen zur Unterstützung der Struktur, beispielsweise für Hintergrundbereiche oder Navigationsleisten.
	\item \textbf{Neutralfarben:} Dezente Farben wie Weiß, Grau und Schwarz werden genutzt, um Inhalte klar und übersichtlich zu präsentieren, ohne die Hauptfarben zu überlagern.
\end{itemize}

Die Übertragung der Logo-Farben in die Benutzeroberfläche sorgt für eine harmonische Verbindung zwischen der Corporate Identity und dem Design des Webshops. Dies unterstützt nicht nur die Ästhetik, sondern stärkt auch die Wiedererkennbarkeit der Marke.
\subsection{Schriften}
Als Schriftart wurde \textit{Roboto} gewählt, da sie modern und gut lesbar ist. Die Schriftgröße und -farbe wurden so angepasst, dass Inhalte auch bei längerer Nutzung gut wahrgenommen werden können.

\subsection{Layout}
Das Layout des Webshops ist klar strukturiert und folgt etablierten Standards:
\begin{itemize}
	\item \textbf{Navigation:} Eine horizontale Hauptnavigation bietet schnellen Zugriff auf Kategorien und wichtige Funktionen.
	\item \textbf{Produktdarstellung:} Produkte werden in einer Rasteransicht präsentiert, um eine übersichtliche Darstellung zu gewährleisten.
	\item \textbf{Interaktive Elemente:} Buttons und Links sind optisch hervorgehoben, sodass sie leicht identifizierbar sind.
\end{itemize}

\section{Verzicht auf Mockups}
Für die Gestaltung der Benutzeroberfläche wurden keine Mockups verwendet. Stattdessen erfolgte die Entwicklung direkt iterativ während der Programmierung. Entscheidungen zur Gestaltung wurden im Team getroffen und kontinuierlich basierend auf Feedback optimiert. Diese Vorgehensweise ermöglichte eine flexible Anpassung an die Projektanforderungen.

\section{Zusammenfassung}
Die Benutzeroberflächen des Webshops wurden mit dem Ziel entwickelt, ein benutzerfreundliches und funktionales Design zu schaffen. Durch die Orientierung an klaren Richtlinien und Standards konnte eine Oberfläche geschaffen werden, die Geschäftskunden ein angenehmes und effizientes Nutzungserlebnis bietet.
\chapter{Nicht-funktionale Anforderungen}
\begin{description}
	\item[NF10] Das Produkt soll plattformunabhängig sein.
	\item[NF20] Das Produkt muss anwenderfreundlich sein (intuitive Bedienbarkeit für Benutzer ohne EDV-Vorkenntnisse, umfangreiche Hilfefunktion)
	\item[NF30] Das Produkt muss mit geringem Aufwand weiterentwickelbar und wartbar sein.
	\item[NF40] Das Produkt soll fehlertolerant bezüglich Bedien- und Eingabefehler sein.
\end{description}

\chapter{Technische Produktumgebung}

In diesem Kapitel wird die technische Umgebung des Webshops beschrieben. Da es sich um eine Client-Server-Anwendung handelt, werden die Umgebungen für den Client und den Server getrennt betrachtet.

\section{Software}
Die Bereitstellung des Webshops erfolgt über einen Virtual Private Server (VPS). Die eingesetzten Softwaresysteme umfassen:
\begin{itemize}
	\item \textbf{Server-Betriebssystem:} Linux (Ubuntu)
	\item \textbf{Dokploy:} Deployplattform für die Orchestrierung und Verwaltung von Servicen
	\item \textbf{Webserver:} NGINX für die Auslieferung des Frontends
	\item \textbf{Backend:} Python Django Framework mit Uvicorn
	\item \textbf{Datenbank:} PostgreSQL für die Speicherung von Produkt-, Bestands- und Kundendaten sowie SQLite für die Testumgebung
	\item \textbf{Containerisierung:} Docker für die Bereitstellung und Isolation der verschiedenen Komponenten
	\item \textbf{Reverse Proxy:} Traefik zur Verteilung und Absicherung der Anfragen
	\item \textbf{Frontend:} Angular für die Entwicklung der Benutzeroberfläche
	\item \textbf{Mailservice:} Mailservice in Go, mit REST-Schnittstelle welche vom Backend getriggert wird.
	\end{itemize}
Für den Client wird lediglich ein browserfähiges Endgerät benötigt. Der Webshop ist mit allen modernen Webbrowsern (z. B. Chrome, Firefox, Edge) kompatibel.

\section{Hardware}
Die minimalen und maximalen Anforderungen an die Hardware sind wie folgt:
\begin{itemize}
	\item \textbf{Server:} VPS mit mindestens 2 CPU-Kernen, 4 GB RAM und 50 GB Speicherplatz
	\item \textbf{Client:} Ein browserfähiges Endgerät mit Internetzugang
\end{itemize}

\section{Orgware}
Der Webshop wird unter den folgenden organisatorischen Randbedingungen eingesetzt:
\begin{itemize}
	\item Der Server benötigt eine stabile Netzwerkverbindung, um alle Anfragen der Clients zu verarbeiten und mit der Datenbank zu kommunizieren.
	\item Der Zugriff auf den Webshop erfolgt über gängige Webbrowser, wodurch keine zusätzliche Softwareinstallation auf Client-Geräten notwendig ist.
	\item Administratoren und Entwickler benötigen Zugriff auf den VPS zur Wartung, Überwachung und Weiterentwicklung des Systems
\end{itemize}

\chapter{Konzeption der KI-Komponente}
\section{Idee}
Für den Webshop soll ein Chatbot erstellt werden. 
Dieser soll Fragen der Kunden beantworten, um die Kundenzufriedenheit zu erhöhen.
\section{Zielgruppenanalyse}
Der KI-Chatbot richtet sich vor allem an IT-Einkäufer, die nach Produkten mit bestimmten technischen Spezifikationen suchen und einzelne Produkte schnell vergleichen wollen.
\section{Funktionen und Anwendungsfälle}
Produktberatung:
\begin{itemize}
	\item Empfehlung geeigneter Produkte auf Basis von Kundenanforderungen (z.~B. \enquote{Ich suche einen Monitor mit mindestens 24 Zoll})
	\item Produktvergleiche (z.~B. \enquote{Was sind die Unterschiede zwischen Laptop A und Laptop B?})
\end{itemize}
\section{Umsetzungsmöglichkeiten}
Es gibt mehrere Möglichkeiten, einen Chatbot zu erstellen:
\begin{enumerate}
	\item Erstellung eines eigenen Large Language Models, welches mit Produktdaten trainiert wird.
	\item Nutzung eines Chatbot-Anbieters, der mit Trainingsdaten aus dem B2B-Webshop arbeitet.
\end{enumerate}
Diese Möglichkeiten werden im Folgenden verglichen.
\subsection{Eigenes Large Language Model (LLM)}
\begin{table}[ht]
	\centering
	\begin{tabularx}{\textwidth}{X|X}
		Vorteile & Nachteile \\ \hline \hline
		\textbf{Maximale Kontrolle} & \textbf{Hoher Entwicklungsaufwand} \\
		$\cdot$ Modell kann individuell trainiert werden & $\cdot$ Aufbau eines eigenen LLMs erfordert erhebliches Know-how\\
		$\cdot$ Vollständige Anpassung der Antworten & $\cdot$ Längere Implementierungszeit\\ \hline
		\textbf{Datenschutz und Sicherheit} & \textbf{Wartung und Aktualisierung} \\
		$\cdot$ Alle Daten bleiben im eigenen System, was im Hinblick auf Datenschutzregelungen vorteilhaft ist & $\cdot$ Kontinuierliche Pflege, Optimierung und Nachtrainieren des Modells sind erforderlich, um relevante Ergebnisse zu liefern \\
		- & $\cdot$ Aufwendige Skalierung bei steigender Nutzung \\
	\end{tabularx}
	\caption{Abwägung der Nutzung eines LLMs}
\end{table}
Ein eigenes LLM ist dann zu empfehlen, wenn der Schutz der Privatsphäre und die vollständige Anpassbarkeit im Vordergrund stehen.
\subsection{Chatbot-Anbieter}
\begin{table}[ht]
	\centering
	\begin{tabularx}{\textwidth}{X|X}
		Vorteile & Nachteile \\ \hline \hline
		\textbf{Schnelle Implementierung} & \textbf{Abhängigkeit vom Anbieter} \\
		$\cdot$ Anbieter bieten oft vorgefertigte Tools, APIs und intuitive Dashboards, die eine schnelle und einfache Integration ermöglichen & $\cdot$ Daten und Geschäftsprozesse werden auf die Plattform des Anbieters ausgelagert, was zu Abhängigkeiten führen kann\\
		$\cdot$ Kein Aufbau eines eigenen Modells notwendig & -\\ \hline
		\textbf{Wartung und Updates} & \textbf{Datenschutzrisiken} \\
		$\cdot$ Anbieter übernimmt Wartung, Optimierung und die Bereitstellung aktueller KI-Modelle & $\cdot$ Daten werden an Drittanbieter übermittelt, die teilweise Server außerhalb der EU verwenden \\
		$\cdot$ Regelmäßßige Updates sorgen dafür, dass das Modell auf dem neuesten Stand der Technik bleibt & - \\ \hline
		\textbf{Skalierbarkeit} & \textbf{Eingeschränkte Anpassbarkeit}\\
		$\cdot$ Anbieter verfügen über skalierbare Infrastrukturen, die automatisch auf steigende Nutzungsanforderungen reagieren können & $\cdot$ Anbieter bieten oft weniger Möglichkeiten, das Modell tiefgehend auf unternehmensspezifische Anforderungen anzupassen\\
		- & $\cdot$ Limitierte Kontrolle über den Trainingsprozess des Modells\\
	\end{tabularx}
	\caption{Abwägung der Nutzung eines externen Chatbots}
\end{table}
Ein Chatbot-Anbieter wird empfohlen, wenn eine schnelle Implementierung und der Zugang zu den neuesten KI-Technologien wichtiger sind.

Im Fall des B2B-Webshops wird der Chatbot über einen Chatbot-Anbieter bereitgestellt, da vor allem die schnelle Implementierung für das zeitlich begrenzte Projekt von Vorteil ist. Die Trainingsdaten stehen erst wenige Wochen vor Projektende zur Verfügung, so dass nur wenig Zeit für die Implementierung bleibt.
\section{Anforderungen an den Chatbot}
\begin{table}[ht]
	\centering
	\begin{tabularx}{\textwidth}{X|X}
		Funktionale Anforderungen&Nicht-funktionale Anforderungen\\ \hline \hline
		$\cdot$ Erkennung von Kundenanfragen in natürlicher Sprache & $\cdot$ Vollständige Antworten in weniger als 10 Sekunden\\
		$\cdot$ Unterstützung mehrstufiger Dialoge & $\cdot$ 24/7-Zugriff\\
		- & $\cdot$ Intuitive Interaktion, auch für nicht-technische Nutzer\\
	\end{tabularx}
	\caption{Anforderungen an den Chatbot}
\end{table}
\section{Datenanforderungen}
Die für den Chatbot erforderlichen Produktdaten sind unter anderem die Artikelnummer, die technischen Spezifikationen und die Preise.
\section{Auswahl des Anbieters}
Nach Recherche und Vergleich mehrerer Anbieter wurde \enquote{Botpress} als Dienstleister ausgewählt. 
Dieser bietet in seinem kostenlosen Modell die besten Möglichkeiten und 500 kostenlose Anfragen pro Monat.
Darüber hinaus besteht die Möglichkeit, den Chatbot an die spezifischen Anforderungen anzupassen und eine reibungslose Integration in das Frontend zu gewährleisten.
\section{Bot-Instructions}
Gemäß den Vorgaben der Bot-Instructions ist der Chatbot dazu angehalten, spezifische Antworten zu geben und festgelegte Zuständigkeiten zu erfüllen. Es wurde eine Erwartung an die Klarheit und Konsistenz der Antworten des Chatbots gestellt, wobei gleichzeitig die Interaktionsmöglichkeit des Nutzers durch Buttons festgesetzt wurde. Darüber hinaus wurde spezifiziert, dass der Bot Produktempfehlungen ausgeben soll, die den Angaben des Kunden entsprechen. 
\section{Datenintegration in Botpress}
Die Daten können Botpress auf verschiedene Weise zur Verfügung gestellt werden. 
Die beiden einfachsten Möglichkeiten sind die Eingabe eines Links zur Website oder die Bereitstellung über eine CSV-Datei.

Bei der ersten Möglichkeit wird die Website bei einer Anfrage analysiert und das Ergebnis an den Kunden zurückgegeben. 
Nach mehreren Tests mit einer Website hat sich herausgestellt, dass diese Methode ungenaue Antworten liefert und nicht immer die richtigen Produkte ausgibt bzw. bei unpräzisen Anfragen keine passenden Antworten zurückgegeben werden.

Die zweite Möglichkeit besteht darin, eine CSV-Datei mit den Produktdaten in Botpress hochzuladen. 
Nach mehreren Tests mit Testdaten hat sich gezeigt, dass die Antworten präziser sind und auch bei ungenauen Anfragen sinnvolle Antworten zurückgegeben werden. 
Ein Nachteil ist, dass die Daten immer manuell aktualisiert werden müssen. 
Dennoch ist diese Methode für den Einsatz im B2B-Webshop besser geeignet, da auch ungenaue Anfragen sinnvoll beantwortet werden müssen.  

Für die Realisierung im Zuge des Projektes wurde aufgrund der vorliegenden Vorteile die zweite Variante gewählt. Die Produktdaten welche im Webshop eingebunden werden, wurde in die Wissensdatenbank des Chat-Bots eingefügt.

\section{Integration des Chatbots}
Nach Integration der Produktdaten bei dem Anbieter und der darauffolgenden Publikation werden seitens Botpress zwei Skript-Dateien bereitgestellt. Diese Skript-Zeilen können anschließend in das Frontend eingebettet werden. Dadurch wird der Chatbot für die Webshop-Besucher sichtbar und kann verwendet werden.

\section{Bearbeitungsablauf einer Kundenanfrage}
Im Falle der Eingabe einer Frage in den Chat erfolgt eine Analyse durch das LLM. Für die Beantwortung einfacher Anfragen wird GPT-4o Mini verwendet, für eine bestmögliche Beantwortung GPT-4o und für die Generierung der Antwort. Nach Analyse wird eine Durchsicht der Knowledge Base durchgeführt, in diesem Fall der Produktdatenbank, da keine weitere Knowledge Base zur Verfügung steht. Nach dem Auffinden der passenden Informationen erfolgt die Formulierung einer Antwort und der bereitstellung für den Anfragenden.


\chapter{Projektorganisation}
\section{Projektmethodik}
Die Entwicklung des Webshops für den \textit{Hot Hardware Hub} erfolgt nach der agilen Methode Scrum, da sie besonders gut für komplexe und dynamische Softwareprojekte geeignet ist. Die Wahl von Scrum basiert auf mehreren entscheidenden Faktoren, die sich sowohl auf die Anforderungen des Projekts als auch auf die Besonderheiten der IT-Hardware-Branche beziehen.

\begin{enumerate}
	\item \textbf{Flexibilität und Anpassungsfähigkeit an Marktveränderungen} \\
	Die IT-Hardware-Branche ist einem schnellen Wandel unterworfen. Technologien entwickeln sich stetig weiter, neue Produkte kommen auf den Markt, und Kundenanforderungen ändern sich kontinuierlich. Eine klassische, langfristig fixierte Projektplanung (z. B. nach dem Wasserfall-Modell) wäre in einem solch dynamischen Umfeld nicht optimal, da sie wenig Spielraum für kurzfristige Anpassungen lässt.
	
	Scrum hingegen ermöglicht eine iterative Entwicklung in kurzen Sprints, sodass regelmäßig neue Funktionen ausgeliefert und Feedback berücksichtigt werden können. Sollte sich beispielsweise herausstellen, dass Geschäftskunden verstärkt eine bestimmte Filterfunktion oder eine alternative Bezahlmethode benötigen, kann dies flexibel in den nächsten Sprint aufgenommen und zeitnah umgesetzt werden.
	
	\item \textbf{Frühzeitige und kontinuierliche Lieferung von Produktinkrementen} \\
	Ein zentraler Vorteil von Scrum ist die Möglichkeit, den Webshop schrittweise zu entwickeln und dabei nach jedem Sprint ein funktionsfähiges Produktinkrement bereitzustellen. Dadurch wird verhindert, dass das gesamte Projekt erst nach Monaten oder Jahren einsatzbereit ist. Stattdessen können erste Versionen – beispielsweise als \textit{Minimal Viable Product} (MVP) – schon früh veröffentlicht und dann fortlaufend verbessert werden.
	
	Diese Vorgehensweise ermöglicht es, den Markt zu testen, erste Kunden zu gewinnen und deren Feedback in die Weiterentwicklung einfließen zu lassen. So kann sichergestellt werden, dass der Webshop von Beginn an kundenorientiert ist und sich an deren tatsächlichen Bedürfnissen ausrichtet.
	
	\item \textbf{Kundenorientierung und Einbindung von Stakeholdern} \\
	Da der \textit{Hot Hardware Hub} speziell für Geschäftskunden konzipiert ist, ist es essenziell, dass deren Anforderungen und Wünsche kontinuierlich in die Entwicklung einfließen. Im Scrum-Framework übernimmt der \textbf{Product Owner} diese Rolle, indem er die Interessen der Kunden vertritt, Anforderungen priorisiert und das \textit{Product Backlog} verwaltet.
	
	Regelmäßige Meetings, wie die \textit{Sprint Reviews}, bieten zudem die Möglichkeit, dass Stakeholder (z. B. zukünftige Nutzer, Geschäftspartner oder interne Entscheider) den Entwicklungsfortschritt begutachten und frühzeitig Rückmeldungen geben. Dadurch wird sichergestellt, dass der Webshop von Anfang an marktgerecht entwickelt wird und spätere, kostenintensive Nachbesserungen vermieden werden.
	
	\item \textbf{Verbesserung der Zusammenarbeit im Entwicklungsteam} \\
	Scrum fördert eine transparente und effiziente Zusammenarbeit innerhalb des Teams. Durch \textit{Daily Scrums} wird sichergestellt, dass jedes Teammitglied stets über den aktuellen Stand der Entwicklung informiert ist. Probleme oder Hindernisse werden frühzeitig erkannt und können schnell gelöst werden.
	
	Zudem sorgen \textit{Sprint Retrospektiven} dafür, dass das Team kontinuierlich aus vergangenen Sprints lernt und seine Arbeitsweise verbessert. Dies steigert nicht nur die Effizienz, sondern auch die Qualität des Endprodukts.
	
	\item \textbf{Risikominimierung durch regelmäßige Tests und Qualitätskontrollen} \\
	Ein weiteres zentrales Argument für die Nutzung von Scrum ist die Minimierung von Risiken durch regelmäßige Tests und Überprüfungen. Da in jedem Sprint ein testbares Produktinkrement entsteht, können Fehler frühzeitig erkannt und behoben werden.
	
	Im Gegensatz zu traditionellen Entwicklungsmodellen, bei denen ein Produkt erst am Ende der Entwicklungsphase umfassend getestet wird, stellt Scrum sicher, dass kontinuierlich getestet wird. Dies reduziert das Risiko schwerwiegender Fehler oder technischer Probleme, die ansonsten erst spät erkannt würden.
	
	\item \textbf{Schnellere Markteinführung und höhere Wettbewerbsfähigkeit} \\
	Durch die iterative Entwicklung und die regelmäßige Auslieferung neuer Funktionen kann der Webshop schneller auf den Markt gebracht werden. Ein früher Release als \textit{MVP} ermöglicht es, bereits erste Umsätze zu generieren und auf Basis von Kundenfeedback weiterzuentwickeln.
	
	Gerade im B2B-Bereich, wo Unternehmen nach effizienten und unkomplizierten Beschaffungslösungen suchen, ist es wichtig, möglichst früh eine funktionierende Plattform bereitzustellen. Durch Scrum kann der Webshop frühzeitig in Betrieb genommen und schrittweise verbessert werden, was einen klaren Wettbewerbsvorteil darstellt.
\end{enumerate}

\section{Rollenverteilung}

Um eine klare Aufgabenverteilung sicherzustellen und die Effizienz des Teams zu maximieren, wurde das Projektteam in verschiedene Verantwortungsbereiche eingeteilt. Nachfolgend wird die Rollenverteilung detailliert beschrieben:

\subsection{Organisatorische Leitung}
Die organisatorische Leitung des Projekts übernehmen \textbf{Colin Christ} und \textbf{Steffen Spatzek}. Ihre Aufgaben umfassen:
\begin{itemize}
	\item Koordination der übergeordneten Prozesse
	\item Führung des Teams und Sicherstellung einer effizienten Zusammenarbeit
	\item Verantwortung für die Kommunikation zwischen den einzelnen Bereichen
	\item Sicherstellung der Zielerreichung und Einhaltung von Deadlines
	\item Eingreifen bei Problemen während der Sprints
	\item Vorbereitung und Durchführung von Terminen (Zwischenstand vor dem Plenum und Dozenten)
	\end{itemize}

\subsection{Organisation}
Die Verantwortung für die organisatorischen Inhalte des Projekts liegt bei:
\begin{itemize}
	\item \textbf{Lukas Tomanek}, \textbf{Colin Christ}, \textbf{Steffen Spatzek}, \textbf{David Schreck}, \textbf{Bastian Bamberger}, \textbf{Milan Jeevakanthan}, \textbf{Luis Väth} und \textbf{Daniel Arnold}
\end{itemize}
Ihre Aufgaben umfassen:
\begin{itemize}
	\item Erstellung und Pflege der Projektdokumentation
	\item Regelmäßige Aktualisierung des Pflichtenhefts
	\item Erstellung von Modellen und Diagrammen
	\item Planung und Bearbeitung der Sprints
	\item Präzise Formulierung der Use-Cases als Grundlage für die Projektarbeit
	\item Finale Version des Pflichtenhefts als LaTex-Dokument
\end{itemize}

\subsection{Entwicklung}
Das Entwicklungsteam setzt sich aus den folgenden Mitgliedern zusammen:
\begin{itemize}
	\item \textbf{Steffen Spatzek}: Führende Rolle in der Entwicklung, verantwortlich für die technische Umsetzung, insbesondere:
	\begin{itemize}
		\item Weiterentwicklung des Django Frameworks
		\item Containerisierung der Services
		\item Verantwortung für den Deployprozess und die VPS-Überwachung
		\item Erstellung und Bereitstellung des E-Mail-Services
		\item Mitwirkung bei der Fehlerbehebung und Verbesserung des Frontends 
	\end{itemize}
	\item \textbf{Alan Kanjo}: Unterstützte maßgeblich bei der Bereitstellung des Django Frameworks und der Containerisierung.
	\item \textbf{Andreas Denz}: Federführend verantwortlich für die Entwicklung des Frontends in Angular und Unterstützung im Backend.
	\item \textbf{Paul Kunz}, \textbf{Julian Strohm} sowie \textbf{David Schreck}: Unterstützten bei verschiedenen Aspekten der Softwareentwicklung, darunter:
	\begin{itemize}
		\item Unterstützung bei Backend- und Frontend-Funktionalitäten
		\item Mithilfe bei der Entwicklung neuer Features für den Webshop
		\item Bereitstellung von HTML-Templates für den E-Mail-Versand. 
		\item Produktdatenaufbereitung und -bereitstellung
	\end{itemize}
\end{itemize}

\subsection{Solution Architecture}
Die technische Architektur des Systems wird von \textbf{Steffen Spatzek} sowie \textbf{Alan Kanjo} verantwortet. Die Aufgaben umfassen:
\begin{itemize}
	\item Sicherstellung einer stabilen, skalierbaren und sicheren IT-Infrastruktur
	\item Reproduzierbare Versionen mithilfe von Docker-Containern
\end{itemize}

\subsection{User Experience (UX)}
Für die Gestaltung und Optimierung der Benutzererfahrung sind die folgenden Teammitglieder zuständig:
\begin{itemize}
	\item \textbf{Alan Kanjo}, \textbf{Lukas Tomanek}, \textbf{David Schreck}, \textbf{Julian Strohm}, \textbf{Bastian Bamberger} und \textbf{Milan Jeevakanthan}
\end{itemize}
Ihre Aufgaben umfassen:
\begin{itemize}
	\item Gestaltung und Verbesserung der Benutzeroberfläche
	\item Durchführung von User-Testing
	\item Implementierung von UX-Optimierungen zur Gewährleistung eines reibungslosen und angenehmen Einkaufserlebnisses für Geschäftskunden
	\item Earbeitung eines Konzeptes für die Integration eines KI-Chatbots sowie Vorbereitungsmaßnahme für die Implementierung
\end{itemize}

\subsection{Zusammenfassung}
Diese Rollenverteilung gewährleistet, dass alle relevanten Aspekte des Webshop-Projekts professionell abgedeckt werden. Das Team kann effizient auf Anforderungen reagieren und Herausforderungen während der Entwicklung meistern.

\end{document}