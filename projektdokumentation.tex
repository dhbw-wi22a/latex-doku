\documentclass[%
	ngerman,
	12pt,
	a4paper
]{scrbook}
\usepackage{graphicx}
\usepackage{babel}
\usepackage{minted}
\usepackage{hyperref}
\usepackage{fontspec}
\setmainfont{Source Sans Pro}
\setmonofont{Cascadia Code}
\usepackage{tcolorbox}

% Definition der Boxen
\newtcolorbox{tipbox}{
	colback=green!5!white,
	colframe=green!75!black,
	fonttitle=\bfseries,
	title=Tipp,
	sharp corners
}

\newtcolorbox{warningbox}{
	colback=red!5!white,
	colframe=red!75!black,
	fonttitle=\bfseries,
	title=Achtung,
	sharp corners
}

\newtcolorbox{examplebox}{
	colback=blue!5!white,
	colframe=blue!75!black,
	fonttitle=\bfseries,
	title=Beispiel,
	sharp corners
}

\title{Pflichtenheft}
\subtitle{Konzeption und prototypische Implementierung eines B2B-Webshops}
\author{WI22A-AKI}

\begin{document}
\maketitle

\section{Projektteilnehmer}
\begin{tabular}{l|l|l}
	\textbf{Name}                & \textbf{Vorname} & \textbf{Matrikelnummer} \\ \hline
	Christ (Projektleiter)       & Colin            & 4359760                 \\
	Spatzek (stv. Projektleiter) & Steffen          & 3854031                 \\ \hline
	Arnold                       & Daniel           & 8627710                 \\
	Bamberger                    & Bastian          & 2923282                 \\
	Denz                         & Andreas          & 5428962                 \\
	Jeevakanthan                 & Milan            & 9892846                 \\
	Kanjo                        & Alan             & 9795498                 \\
	Kunz                         & Paul             & 2338290                 \\
	Schreck                      & David            & 3533132                 \\
	Strohm                       & Julian           & 7956706                 \\
	Swoboda                      & Timo             & 4388948                 \\
	Tomanek                      & Lukas            & 5985858                 \\
	Väth                         & Luis             & 8122258                 \\
	Weis                         & Noah             & 1555500
\end{tabular}

\tableofcontents

\chapter{Zielsetzung}
Beschreibung des Projektziels:
\begin{itemize}
	\item Warum wird das Projekt durchgeführt?
	\item Welche Probleme sollen gelöst werden?
\end{itemize}

\section{Musskriterien}
	\subsection{Allgemeine Anforderungen}
		\subsubsection{Benutzerverwaltung}
			Rollenbasierte Verwaltung:
			\begin{itemize}
				\item Administrator (alle Rechte, Pflege des Artikelkatalogs)
				\item Interne Mitarbeiter (eingeschränkte Rechte)
				\item Externe Kunden (Anlegen/Pflegen von Accounts, ggf. Sperren)
			\end{itemize}
		\subsubsection{Artikelsuche und Anzeige}
			\begin{itemize}
				\item Katalog zur Suche
				\item semantische Suche
			\end{itemize}
		\subsubsection{Bestellprozess}
		\begin{itemize}
			\item Implementierung eines rudimentären Prozesses für Bestellungen
		\end{itemize}
		\subsubsection{KI-Komponente}
		\begin{itemize}
			\item für Produktempfehlungen oder kundenspezifische Preisfestlegung
		\end{itemize}
		\subsubsection{Aufgabenaufteilung}
		\begin{itemize}
			\item Entwurf sinnvoller Use-Cases, Datenmodelle und Software-Architekturkomponenten
		\end{itemize}
		\subsubsection{Meilensteine}
		\begin{itemize}
			\item Definition von Projektmeilensteinen und Kommunikation bei Abweichungen
		\end{itemize}
		\subsubsection{Software-Engineering-Prinzipien}
		\begin{itemize}
			\item Anwendung von Kern- und Unterstützungsprozessen des Software-Engineering
			\item Einhaltung gängiger Namenskonventionen (z.~B. CamelCase, Methodennamen als Verben)
		\end{itemize}
		\subsubsection{Dokumentation}
		\begin{itemize}
			\item Angemessene und fortlaufende Dokumentation, einschließlich Programmcodes
		\end{itemize}
		\subsubsection{Projektmanagement}
		\begin{itemize}
			\item Regelmäßige Fortschrittsmitteilungen (z.~B. via E-Mail und in einem Log)
		\end{itemize}

\subsection{Technische Anforderungen}
\subsection{Entwicklungsprozess}
\subsection{Präsentation und Abgabe}

\section{Wunschkriterien}
\section{Abgrenzungskriterien}

\chapter{Produkteinsatz}
\section{Anwendungsbereich}
\section{Zielgruppen}
\section{Betriebsbedingungen}

\chapter{Produktübersicht}

\chapter{Detaillierte Produktfunktionen}
\section{User Stories}
\section{Funktionale Anforderungen}
\section{Nicht-funktionale Anforderungen}

\chapter{Produktdaten}
\chapter{Systemarchitektur}
\chapter{Datenmodell}
\chapter{Schnittstellendefinition (API)}
\chapter{Benutzungsoberflächen}

\chapter{Projektorganisation}
\section{Projektmethodik}
\section{Rollenverteilung}

\end{document}