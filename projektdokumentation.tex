\documentclass[%
	ngerman,
	12pt,
	a4paper,
	oneside,
	parskip=full
]{scrbook}
\usepackage{graphicx}
\usepackage{babel}
\usepackage{minted}
\usepackage{hyperref}
\usepackage{fontspec}
\setmainfont{Source Sans Pro}
\setmonofont{Cascadia Code}
\usepackage{tcolorbox}
\usepackage{csquotes}
\usepackage{setspace}

% Definition der Boxen
\newtcolorbox{tipbox}{
	colback=green!5!white,
	colframe=green!75!black,
	fonttitle=\bfseries,
	title=Tipp,
	sharp corners
}

\newtcolorbox{warningbox}{
	colback=red!5!white,
	colframe=red!75!black,
	fonttitle=\bfseries,
	title=Achtung,
	sharp corners
}

\newtcolorbox{examplebox}{
	colback=blue!5!white,
	colframe=blue!75!black,
	fonttitle=\bfseries,
	title=Beispiel,
	sharp corners
}

\title{Pflichtenheft}
\subtitle{Konzeption und prototypische Implementierung eines B2B-Webshops}


\author{WI22A-AKI}


\begin{document}
%\maketitle

\begin{titlepage}
	\centering
	% Logo hinzufügen
	\includegraphics[width=0.5\textwidth]{Media/dhbwlogo.png}\vspace{1cm}

	{\large Projektkonzeption und -realisierung\par}
	\vspace{0.5cm}
	{\Large\bfseries Konzeption und prototypische Implementierung eines B2B-Webshops\par}
	\vspace{1.5cm}

%	{\Huge \texttt{\MakeUppercase{Pflichtenheft}}}\par
	{\Huge \textbf{Pflichtenheft}\par}
	\vspace{2cm}

	\doublespacing
	{\large \textbf{Wintersemester 2024/2025}\par}
	\textbf{Semester:} 5\\
	\textbf{Kurs:} WI22A-AKI

	\vspace{2cm}

	{\large\textbf{Betreuer:}\par}
	{\large Prof. Dr. Alexandros Nanopoulos} \\
	{\large Prof. Dr. Dirk Palleduhn}

	\vfill

	Mosbach, den \today  %TODO

\end{titlepage}


\section{Projektteilnehmer}
\begin{tabular}{l|l|l}
	\textbf{Name}                & \textbf{Vorname} & \textbf{Matrikelnummer} \\ \hline
	Christ (Projektleiter)       & Colin            & 4359760                 \\
	Spatzek (stv. Projektleiter) & Steffen          & 3854031                 \\ \hline
	Arnold                       & Daniel           & 8627710                 \\
	Bamberger                    & Bastian          & 2923282                 \\
	Denz                         & Andreas          & 5428962                 \\
	Jeevakanthan                 & Milan            & 9892846                 \\
	Kanjo                        & Alan             & 9795498                 \\
	Kunz                         & Paul             & 2338290                 \\
	Schreck                      & David            & 3533132                 \\
	Strohm                       & Julian           & 7956706                 \\
	Swoboda                      & Timo             & 4388948                 \\
	Tomanek                      & Lukas            & 5985858                 \\
	Väth                         & Luis             & 8122258                 \\
	Weis                         & Noah             & 1555500
\end{tabular}

\tableofcontents

\chapter{Zielsetzung}
	Der \textbf{Hot Hardware Hub} ist ein fiktives Unternehmen, das im Rahmen dieses Projekts gegründet wurde, um hochwertige und moderne IT-Hardware speziell für Geschäftskunden (B2B) anzubieten. Ziel ist es, einen innovativen Webshop zu entwickeln, der es Unternehmen ermöglicht, ihre benötigte Hardware schnell, einfach und bequem online zu bestellen.
	Im Mittelpunkt des Systems steht das Ziel, den Kunden ein erstklassiges Einkaufserlebnis zu bieten. Eine intuitive Benutzeroberfläche sowie ein klar strukturierter und gut durchsuchbarer Produktkatalog ermöglichen es den Nutzern, mit nur wenigen Klicks die passenden Produkte zu finden und ihre Bestellungen mühelos abzuschließen.
	Die Systementwicklung reagiert auf die wachsende Nachfrage nach digitalen Beschaffungslösungen im IT-Bereich. Viele Unternehmen suchen nach effizienten Möglichkeiten, hochwertige Hardware zeitnah und unkompliziert zu erwerben. Der Webshop des \textbf{Hot Hardware Hub} stellt hierfür eine zuverlässige und benutzerfreundliche Plattform bereit, die die IT-Beschaffung deutlich erleichtert. Das Ziel ist es, nicht nur Zeit und Aufwand zu sparen, sondern auch die Zufriedenheit und Effizienz der Geschäftskunden nachhaltig zu steigern.

\section{Musskriterien}
	\vspace{0.5cm}
	\textbf{Produktkatalog}
	\begin{itemize}
		\item Kunden können den Produktkatalog mit der IT-Hardware einsehen.
		\item Produkte werden in Kategorien darstellbar angezeigt.
		\item Produkte müssen die wesentlichen Merkmale (Preis, Menge, Warenbestand und Produktdetails) dem Kunden sichtbar machen.
		\item Produkte können über eine Suchfunktion mit Filtermöglichkeiten gezielt gefiltert werden.
	\end{itemize}

	\vspace{0.5cm}
	\textbf{Benutzerverwaltung}
	\begin{itemize}
		\item Kunden müssen sich registrieren und sich mit ihren Daten anmelden können.
		\item Kunden müssen ihre Benutzerdaten einsehen und verändern können.
		\item Kunden müssen ihr Konto deaktivieren bzw. löschen können.
		\item Der Anmeldeprozess im Webshop muss mit einer sicheren Authentifizierungsmethode gestaltet werden.
	\end{itemize}

	\vspace{0.5cm}
	\textbf{Bestellprozess}
	\begin{itemize}
		\item Kunden müssen ihre Produkte in den Warenkorb legen und diesen einsehen können.
		\item Der Kunde muss eine Bestellübersicht vor dem finalen Abschluss sehen.
		\item Der Kunde muss in minimalen Schritten zum erfolgreichen Kaufabschluss geführt werden.
		\item Kunden müssen im Kundenbereich getätigte Bestellungen und deren Status einsehen können.
	\end{itemize}

	\vspace{0.5cm}
	\textbf{Zahlung und Rechnungsstellung}
	\begin{itemize}
		\item Dem Kunden müssen gängige Zahlungsmethoden verfügbar gemacht werden (z.B. Rechnung).
		\item Der Kunde muss nach erfolgreichem Abschluss eine Rechnung per E-Mail erhalten oder diese im Kundenbereich einsehen können.
	\end{itemize}

	\vspace{0.5cm}
	\textbf{Shop-Betreiber}
	\begin{itemize}
		\item Die Admins können Produkte und zugehörige Daten über ein Dashboard erstellen, bearbeiten und löschen.
		\item Die Shop-Betreiber können Produktbilder und -dokumente an die Produktseiten verknüpfen.
	\end{itemize}

	\vspace{0.5cm}
	\textbf{KI-Komponente}
	\begin{itemize}
		\item Es soll eine KI-Komponente eingebettet werden, die dem Kunden beim Einkauf per Chat behilflich ist.
		\item Über eine ML-Komponente soll erreicht werden, dass Kundenpreise individuell rabattiert werden, je nach Einkaufsvolumen bzw. -verhalten.
	\end{itemize}

	\vspace{0.5cm}
	\textbf{Technische Aspekte}
	\begin{itemize}
		\item Der Webshop soll plattformunabhängig von den gängigsten Geräten aufgerufen werden können.
		\item Der Webshop soll Anfragen schnell abarbeiten und schnell erreichbar sein.
		\item Die Webapplikation soll eine intuitive Bedienung aufweisen.
		\item Der B2B-Shop soll durch steigende Produktmengen schnell skalierbar sein.
	\end{itemize}

\section{Wunschkriterien}
	\vspace{0.5cm}
	\textbf{Benutzerverwaltung}
	\begin{itemize}
		\item Es soll ermöglicht werden, dass eine eigene Einkaufsgruppe für einen gewissen Kundenkreis erstellt werden kann.
		\item Der Kunde soll mehr als nur einen Warenkorb anlegen, befüllen und speichern können.
		\item Mehrere Benutzerkonten oder -gruppen für ein Unternehmen sollen unterstützt werden.
		\item Kennzahlen für einen gewissen Kunden bzw. für eine Einkaufsgruppe sollen bereitgestellt werden.
		\item Zwei-Faktor-Authentifizierung oder eine No-Password-Authentication soll dem Kunden ermöglicht werden.
	\end{itemize}

	\vspace{0.5cm}
	\textbf{Produktkatalog}
	\begin{itemize}
		\item Kundenbenachrichtigungen bei wieder verfügbaren Artikeln.
		\item Eine noch detailliertere Filterfunktion bei der Produktsuche.
		\item Produktvergleich-Funktion zwischen zwei oder mehreren Produkten.
		\item Zeitlich begrenzte Aktionen oder individuelle Coupons.
	\end{itemize}

	\vspace{0.5cm}
	\textbf{Sicherheitsaspekte}
	\begin{itemize}
		\item Das System soll alle Aspekte der DSGVO einhalten.
		\item Es sollte eine Datenverschlüsselung gemäß State-of-the-art verwendet werden.
	\end{itemize}

	\vspace{0.5cm}
	\textbf{Bestellprozess}
	\begin{itemize}
		\item Ein wiederkehrendes Bestellmodell soll angeboten werden.
		\item Individuelle Mengenrabatte je nach Menge oder Einkaufsvolumen in einem bestimmten definierten Zeitraum.
	\end{itemize}

	\vspace{0.5cm}
	\textbf{KI-Komponente}
	\begin{itemize}
		\item KI-gestützte Produktempfehlungen basieren auf Wunschlisten, Kaufhistorien oder neuen Artikeln im Sortiment.
	\end{itemize}

	\vspace{0.5cm}
	\textbf{Technische Aspekte}
	\begin{itemize}
		\item Darstellung von Daten wie Traffic, Besucheranzahl und Kundenaktionen in einem Dashboard für die Shopbetreiber.
		\item Logging und Monitoring des Webshops.
	\end{itemize}


\section{Abgrenzungskriterien}
\begin{enumerate}
	\item Funktionale Abgrenzungen:
	\begin{enumerate}
		\item \textbf{Umfang des Produktangebots:} Der Shop beschränkt sich auf Hardwareprodukte, keine Dienstleistungen.
		\item \textbf{Kein Marktplatzmodell:} Der Shop dient nicht als Plattform für andere Anbieter.
	\end{enumerate}
	\item Technische Abgrenzungen:
	\begin{enumerate}
		\item \textbf{Keine mobile Anwendung:} Es wird keine App entwickelt. Der Shop soll als Webservice genutzt werden.
		\item \textbf{Keine Mehrsprachigkeit:} Der Shop wird ausschließlich in deutscher Sprache betrieben.
	\end{enumerate}
	\item Rechtliche Abgrenzungen:
	\begin{enumerate}
		\item \textbf{Keine rechtliche Anpassung für Nicht-EU-Länder:} Der Shop wird nicht an Steuer- und Rechtssysteme außerhalb der EU angepasst.
	\end{enumerate}
	\item Gestalterische Anpassung:
	\begin{enumerate}
		\item \textbf{Keine vollständige Barrierefreiheit:} Der Shop wird nicht vollständig barrierefrei entwickelt (z.~B. keine Optimierung für Screenreader oder spezielle Kontrasteinstellungen).
	\end{enumerate}
\end{enumerate}

\chapter{Produkteinsatz}
\section{Anwendungsbereich}
Der Anwendungsbereich des Webshops umfasst den Verkauf von IT-Hardware an Geschäftskunden.
Die Kunden erhalten Zugang zum Shop und können dort die benötigte Hardware bestellen.
\section{Zielgruppen}
Der B2B-Onlineshop für IT-Hardware richtet sich vor allem an drei Hauptzielgruppen:
IT-Dienstleister, Großunternehmen und Konzerne sowie Reseller.
\begin{itemize}
	\item IT-Dienstleister und Systemhäuser benötigen regelmäßig Hardware wie Server, Netzwerktechnik und Speichersysteme für den Aufbau und die Wartung von IT-Infrastrukturen bei ihren Kunden.
	Diese Zielgruppe verlangt große Bestellmengen, maßgeschneiderte Lösungen und eine zuverlässige Lieferung.
	\item Große Unternehmen und Konzerne beschaffen IT-Hardware für ihre Mitarbeiter und Abteilungen.
	Sie benötigen eine breite Produktpalette und einfache Bestellprozesse.
	\item Reseller hingegen kaufen IT-Produkte in großen Mengen, um sie weiterzuverkaufen.
	Sie benötigen wettbewerbsfähige Preise, detaillierte Produktinformationen sowie eine effiziente Bestell- und Lieferabwicklung.
\end{itemize}
\section{Betriebsbedingungen}
Die Anwendung läuft auf einem Webserver in einer eigenen containerisierten Docker-Umgebung.
Sie wird rund um die Uhr laufen, mit Ausnahme von Wartungsarbeiten.

\chapter{Produktübersicht}
%TODO
Einbetten der Use-Case-Diagramme mit den diversen Akteuren (Akteursbeschreibung für Kunde, Shopmitarbeiter, Admins)

Ggf. Produktfunktionen anhand des Diagramms einbetten und dann untergliedern und mit Literalen/Ziffern zur besseren untergliederung versehen -
Beispiel:

F.1 Produktkatalog
F1.1 Kunde kann Produktkatalog ansehen
F1.2 Kunden kann im Produktkatalog nach Suchkriterien suchen
F1.3 Kunde kann per Filterfunktion gewisse Artikel ausblenden/einblenden
F1.4 Kunde kann gezielt ein Produkt in der Detailansicht öffnen


\section{Katalog- und Kategoriestruktur}
\begin{itemize}
	\item \textbf{Hierarchie:} Die Produkte sind in Kategorien und Unterkategorien gruppiert, z.~B. \enquote{Elektronik $\rightarrow$ Bauteile $\rightarrow$ Widerstände}
	\item \textbf{Filterfunktionen:} Kunden können Produkte nach Merkmalen, wie Preis, Verfügbarkeit, Marke oder Spezifikationen filtern
	\item \textbf{Navigation:} Intuitive Benutzerführung erleichtert das Auffinden bestimmter Produkte
\end{itemize}
\section{Produktdetails auf einen Blick}
\begin{itemize}
	\item \textbf{Produktname:} Klar und eindeutig, idealerweise mit Artikelnummer
	\item \textbf{Kurzbeschreibung:} Wichtige Eigenschaften oder Anwendungsbereiche des Produkts
	\item \textbf{Bilder}
	\item \textbf{Preisangaben:} Nettopreise für B2B, ggf. Staffelpreise oder Rabatte
	\item \textbf{Verfügbarkeitsstatus:} Angaben zum Lagerbeständen oder Lieferzeiten
\end{itemize}
\section{Technische Daten und Spezifikationen}
\begin{itemize}
	\item Für B2B-Kunden sind detaillierte technische Informationen oft entscheidend (z.~B. Material, Abmessungen, Zertifizierungen)
	\item Datenblätter oder technische Zeichnungen zum Herunterladen
\end{itemize}
\section{Individuelle Kundenanforderungen}
\begin{itemize}
	\item \textbf{Personalisierte Preise:} Preise auf Basis von Kundenverträgen oder Mengenrabatten
	\item \textbf{Bestellhistorie:} Möglichkeit, bereits gekaufte Produkte erneut zu bestellen
	\item \textbf{Vergleichsfunktionen:} Direkter Vergleich mehrerer Produkte
\end{itemize}
\section{Interaktive und UX-Elemente}
\begin{itemize}
	\item \textbf{Responsive Design:} Optimierung für verschiedene Geräte bzw. Oberflächen
	\item \textbf{Schnellsuche:} Vorschläge und Autovervollständigung bei Eingabe von Suchbegriffen
	\item \textbf{API-Integration:} Erlaubt Kunden, die Produktdaten direkt in ihre internen Systeme zu integrieren
\end{itemize}

\chapter{Detaillierte Produktfunktionen}
\section{User Stories}
Statt doppelter Dokumentation, nur auf das Scrum-Board verweisen: \\
\url{https://tree.taiga.io/project/ssptzk-b2b-webshop}
\newline
%TODO
Bedarf einer Abklärung mit den Betreuern am Freitag 05.12.2024

\section{Funktionale Anforderungen}
Eingliederung in Geschäftsfälle gängige Praxis, siehe Beispiele in Discord

Siehe Kap 4 - \url{https://www.ibr.cs.tu-bs.de/courses/ss07/sep-cm/templates/pflichtenheft.pdf}


\chapter{Produktdaten}
\textbf{Beispiel} \\
Artikeldaten (max. 50):\\
Bezeichnung, eine kurze Beschreibung, eine ausführliche Beschreibung, Preise
/D10/
(Einkaufspreis beim Hersteller, empfohlener Verkaufspreis des Herstellers,
tatsächlicher Verkaufspreis im Markt), weitere Herstellerinformationen (u.U.
komplexes Datum)
Bestandsdaten (max. 500):\\
im Markt zum Verkauf zur Verfügung stehende Anzahl der Artikel des Warenkatalogs,
/D20/
Produktanzahlen in den Einkaufskörben der Kunden, Anzahlen der bestellten Artikel in
den ausstehenden Lieferungen
/D30/ Personaldaten (max. 50):\\
Name, Vorname, Einstellungsdatum, Einsatz-(Aufgaben-)bereich, Gehalt
/D40/ Kundendaten (max. 50):\\
Name, Vorname, Lieferadresse für externe Lieferungen (komplexes Datum), bisheriges
Einkaufsvolumen zur Berechnung des Treuerabattes, sein Einkaufskorb (wenn er im
Markt ist) (komplexes Datum)

\chapter{Qualitätsanforderung}
In diesem Kapitel wird festgelegt, welche Qualitätsmerkmale das zu entwickelnde Produkt in
welcher Qualitätsstufe besitzen soll. Vorraussetzung ist, dass Qualitätsmerkmale in
operationalisierter Form vorliegen. Die operationalisierten Qualitätsmerkmale sind als
Anhang dem Pflichtenheft beizufügen, falls sie nicht als allgemeine Richtlinie (z. B. Standard,
Norm) zur Verfügung gestellt werden können.

Als Beispiel folgende Kriterien:

\begin{itemize}
	\item Funktionalität
	\item Angemessenheit
	\item Richtigkeit
	\item Interoperabilität
	\item Ordnungsmäßigkeit
	\item Sicherheit
	\item Zuverlässigkeit
	\item Reife
	\item Fehlertoleranz
	\item Wiederherstellbarkeit
	\item Benutzbarkeit
	\item Verständlichkeit
	\item Erlernbarkeit
	\item Bedienbarkeit
	\item Effizienz
	\item Zeitverhalten
	\item Verbrauchsverhalten
	\item Änderbarkeit
	\item Analysierbarkeit
	\item Modifizierbarkeit
	\item Stabilität
	\item Prüfbarkeit
	\item Übertragbarkeit
	\item Anpassbarkeit
	\item Installierbarkeit
	\item Konformität
	\item Austauschbarkeit
\end{itemize}

\chapter{Systemarchitektur}
\begin{itemize}
	\item Systemdiagramm erstellen
	\item Beschreibung Microservices
	\item Übersicht Technologien, Frameworks, Tools
\end{itemize}

\chapter{Datenmodell}
\begin{itemize}
	\item Beschreibung Datenstruktur und deren Beziehung (ER-Diagramm erstellen)
	\item Datenbanktabellen beschreiben, Felder der Tables
\end{itemize}

\chapter{Schnittstellendefinition (API)}
Detaillierte Aufstellung der APIs mit URI, Eingabe- sowie Ausgabeparameter

\chapter{Benutzungsoberflächen}
\begin{itemize}
	\item Erläuterung nach welchen Richtlinien wir uns richten
	\item Mockups einfügen
	\item Verwendete Farben, Schriften, Layout etc.
\end{itemize}

\chapter{Nicht-funktionale Anforderungen}
Es werden alle Anforderungen aufgeführt, die sich nicht auf die Funktionalität, die Leistung
und die Benutzeroberfläche beziehen.
Bitte die Darstellung gemäß Beispiel verwenden.\\
Beispiele:\\
/NF10/ Das Produkt soll plattformunabhängig sein \\
/NF20/ Das Produkt muss anwenderfreundlich sein (intuitive Bedienbarkeit für Benutzer ohne\\
EDV-Vorkenntnisse, umfangreiche Hilfefunktion)
/NF30/ Das Produkt muss mit geringem Aufwand weiterentwickelbar und wartbar sein\\
/NF40 / Die Produkt soll fehlertolerant bezüglich Bedien- und Eingabefehler sein\\


\chapter{Technische Produktumgebung}
%TODO
In diesem Kapitel wird die technische Umgebung des Produktes beschrieben. Bei Client /
Server - Anwendungen ist die Umgebung jeweils für Client und Server getrennt anzugeben.
10.1 Software\\
Hier wird angegeben, welche Softwaresysteme (z. B. Betriebssystem, Datenbank,
Fenstersystem, usw.) zur Verfügung stehen.\\
Beispiel:
Server-Betriebssystem: Linux.
Client-Betriebssystem: Windows XP oder Browser (für Fernwartung).
10.2 Hardware \\
Hier werden die Hardware Komponenten (z. B. CPU, Peripherie) in minimaler und maximaler
Konfiguration aufgeführt, die für den Produkteinsatz vorgesehen sind.
Beispiel:\\
Server: PC
Client: PC und browserfähiges Gerät mit Grafikbildschirm (für Fernwartung).
10.3 Orgware\\
Hier wird aufgeführt, unter welchen organisatorischen Randbedingungen bzw.
Voraussetzungen das Produkt eingesetzt werden soll.
Beispiel:\\
Netzwerkverbindung des Servers zum Computersystem der Testmaschinen, von dem die
Abmeldung der Reifen nach durchgeführtem Testlauf kommt.

\chapter{Projektorganisation}
\section{Projektmethodik}
Beschreibung, wieso nach agiler Methode (Scrum) vorgegangen wird
\section{Rollenverteilung}
Rollenverteilung innerhalb des Projektes mit Verantwortlichkeiten darstellen

\end{document}